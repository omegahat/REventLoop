\documentclass{article}
\usepackage{times}
\usepackage{fullpage}
\input{WebMacros}
\input{SMacros}
\input{CMacros}

\author{Duncan Temple Lang}
\title{A Dynamic, run-time Specifiable Event Loop for R}
\begin{document}
\maketitle
\begin{abstract}

  This is a note on the R event loop and a suggestion for a more
  general framework which allows R to be driven by other event loops
  such as Gtk, and Tcl/Tk, Qt, Carbon and CORBA, while still providing
  its existing functionality.  Essentially, this allows us to replace
  the standard R event loop with another, more standard one and add
  the R event sources to that event loop.  This aids greatly in
  embedding R in other applications and also makes integration of
  these other event-driven environments into R both easier and more
  responsive.  We can also use these event loops to provide the basic
  mechanism for asynchronous events to R, including timeouts and
  readers on file descriptors without having to reimplement them
  ourselves.

  In this package, we provide implementations of 
  a few different ideas.
  \begin{enumerate}
  \item allow either the Tcl/Tk or Gtk event loop to be 
  run ``on top'' of the R event loop;
  \item   make the readline mechanism in R
  a regular event driven callback rather than building any knowledge
  of it into the R event loop.
  \item process Tcl/Tk events as part of the Gtk event loop
   and vice-versa. (This has temporarily ceased working. This may be
   related to changes in the tcltk package and some internal changes
   to the R event loop or it may be a coincidence in the timing.)
  \item   an abstract data structure for a generic event
    loop with implementations for both Gtk and Tcl/Tk.
  \end{enumerate}
 

  We outline a mechanism by which this general
  event loop structure can be defined and integrated directly into R.

  A specific example of what this mechanism provides is the
  integration of GGobi, its background tour motion, tooltips from Gtk
  applications, running Tcl GUIs within this environment as well as
  dynamic resizing graphics devices all while R is ``blocked'' in
  either a sleep or a call to, e.g. \SFunction{scan}.
  
  This mechanism currently applies only to Unix-like environments.
  Soon we will explore integrating this with Windows and OS X.

\end{abstract}



\section{Gtk+}

We start by looking at how we can use the Gtk+ event loop to run R.
Many of the basic steps and concepts map directly to Tcl/Tk and
other event loops.
The least intrusive way to do this (in terms of modified code in the R
tree) involves stacking or running the Gtk+ event loop on top of R's
existing event loop. This essentially involves starting R in the usual
way and invoking a call to a C routine that itself calls the 
event loop \Croutine{gtk_main}.  This will block indefinitely,
processing all event sources known to the Gtk event loop.  For this to
work transparently, we must setup the same user input reader that is
running in R and make it known to the Gtk+ event loop.
% Specifically, we should use the \CVariable{ptr_R_ReadConsole}. 
Before we do this, we can just supply
a readline-based input mechanism.  To do this, we extract the
fundamental readline mechanism used in R and put it in
\file{gtkReader.c}. Then, we register the standard input file
descriptor with the Gtk+ event loop (using
\Croutine{gtk_input_add_full}) and place an appropriate callback to
get readline to handle the input:
\begin{verbatim}
   gtk_input_add_full(fileno(stdin), GDK_INPUT_READ, gtkReadlineHandler, 
                       NULL, NULL, NULL);
\end{verbatim}

In addition to the standard input, we must also ensure that other
event sources currently being monitored by R are also registered with
the Gtk+ event loop. To do this, we loop over the \Ctype{InputHandler}
linked list and register them using
\begin{verbatim}
  gtk_input_add_full(handler->fileDescriptor, GDK_INPUT_READ, handler->handler, 
                      NULL, handler->userData, NULL);
\end{verbatim}


Now, we are almost ready to start handling events.  However, we must
ensure that any errors in top-level evaluations return us to this new
event loop.  For this, we create a top-level context handler using R's
internal mechanism.


\SFunction{setReader} and \SFunction{setMonitor} now need to go to
this event loop.  \Croutine{Rf_AddInputHandler} needs to add to the
existing event loop, not just R's.  For this, we need pointers to
routines that parameterize the event loop.


\subsection{Other Interactive Input}
Browser and scan are two examples of functions that can query the user
for input.  Sleep also needs to be handled carefully so as to allow
other events but block execution of the evaluator and the R engine at
the current spot.  In each of these cases, a recursive/nested event
loop can be used. We arrange to use the currently registered event
loop mechanism but with a callback registered which terminates that
loop when the specific event of interest occurs. 


\SFunction{parse} and other functions call \Croutine{Rstd_ReadConsole}
which is extended to handle this type of event loop.  (See the routine
in \file{gtkReader.c} of the same name.)


\subsection{Terminating the event loop \& the prompt}

When \Croutine{gtk_main_quit} is called within this event loop (and
not a nested one), we terminate this event loop and fall back to the
top-level R event loop.

Unfortunately, we get an extra emission of the prompt after calling
\Croutine{gtk_main_quit}.  For the moment, we insist that the
interactive user call the C routine \Croutine{R_gtk_main_quit} which
sets a flag to indicate the prompt should not be emitted. In the
future, we would like to be able to trap this more elegantly.  But
since it doesn't matter if \Croutine{gtk_main_quit} is called
non-interactively, i.e. as part of an event handler, etc.  the point
is of marginal relevance.


\section{A General Event Loop}
In order to be able to handle different event loops, we need to
identify the different components of the different event loops we may
use.  (In the future we may need to extend this, and again a good
object system even in C would help greatly.)

The essential methods that we need are as follows
(and listed in the \CStruct{R_EventLoop}
definition in \file{Reventloop.h} in this package).

\begin{description}
\item[init] A way to initialize the system such as \Croutine{gtk_init}
  or a routine that calls both \Croutine{Tcl_init} and
  \Croutine{Tk_Init}.

\item[main] This is a routine that is a blocking event loop.  It
  processes all events until it is explicitly terminated by a call to
  the \Croutine{quit} method.  In Gtk, this is \Croutine{gtk_main}.
In Tcl/Tk, it is simply
\begin{verbatim}
 while(Tcl_DoOneEvent(TCL_ALL_EVENTS)) {
     if(tclEventLoopQuit) {
        tclEventLoopQuit = FALSE;
        break;
     }
 }
\end{verbatim}
\item[quit]
This is the routine that terminates the
\textit{current} blocking event loop.

\item[non-blocking iteration]
We also provide a routine for processing
a single event from the event queue
iff there is one available.
This routine should return immediately
(i.e. be non-blocking) if there is no
event in the queue.

\item[add/remove timeout] These allow one to register and unregister
  an event (or specifically a call to a routine) scheduled at a
  particular time interval from now.
  Since Tcl/Tk returns a pointer to a struct,
  while Gtk returns a integer identifier
  to allow one to unregister this, we need
  to use a common base type.
  We use a \CStruct{SEXP} and consider its contents
  as opaque, known only to the event loop system
  itself.

\item[add/remove idle]

\item[add/remove input source]

\end{description}



\section{Event-Driven Programming}

\subsection{Sys.sleep}
Let's take a look at \SFunction{Sys.sleep}.  Rather than simply
calling the system level \Croutine{sleep} (or using a call to
\Croutine{select} as a time-out), we can use either of Gtk or Tcl/Tk's
event loop facilities to specify a timeout for the desired sleep
period. If we simply did this and returned, we wouldn't have attained
the desired effect. Instead, we want to block the R evaluator until
the timeout callback has been invoked.  To get this blocking, while
still processing all the other event sources in the event loop, we
simply create a new sub-event loop.  In the case of Gtk, we call
\Croutine{gtk_main} and this serves exactly that purpose.

In order to exit this event loop at the right time, we arrange to call
\Croutine{gtk_main_quit} after the specified period.
This is easily done by the timeout handler function.
\begin{verbatim}
gint
Rf_gtkMainQuit(gpointer unused)
{
  gtk_main_quit();
  return(0); /* Don't reset this timeout. */
}

SEXP
R_sleep(SEXP interval)
{
  gtk_timeout_add(asInteger(interval), Rf_gtkMainQuit, NULL);
  gtk_main();
  return(R_NilValue);  
}
\end{verbatim}

\subsection{locator}
The function \SFunction{locator} is used to interactively identify
positions in a graphics device.  Currently, this is left to each
graphics device to implement and each is responsible for running a
sub-event loop and filtering out the mouse click events itself.  While
most process all events in their known event queue, their definition
of the event queue is quit narrow.  For instance, the X11 device
processes both X and R event sources.  Similarly, the Gtk device
driver handles Gtk and R events.  (What happens with tkrplot? It is
not a device of this type, it seems from a very, very quick glance.)
Under this model, an X11 driver will not process Gtk idle or timeout
events. So tooltips and other things won't work while we are in a call
to \SFunction{locator} on a different device type from the GUI.

How can we simplify this.  Clearly, if we had a global event loop
object which had a method for handling all events, we could call it.
This is clearly desirable and would solve the issue of handling the
diverse event sources from different ``systems'' (e.g. Tcl/Tk, X and
Gtk). But the way the \SFunction{locator} event loops are structured
requires that they can acess the events directly in the queue and look
at the relevant ones in detail. Specifically, they need to look at
each ``Button Press'' event and examine the x and y values it
contains.

Instead, we can create a callback specifically for button events.
When \SFunction{locator} is called, we register this callback for
these events and enter a new level of the basic event loop.  The
user-level data for this callback contains information about the
number of entries to read (possibly just a single value) and the
arrays into which to put the $(x,y)$ pairs.  When this callback
determines that it has read enough entries (as specified by the
\SArg{n} argument to \SFunction{locator}), it breaks out of this event
loop (via a call to, e.g. \Croutine{gtk_main_quit}) and so
\SFunction{locator} returns.

This is already the way the Gtk device works and many modal dialogs,
etc.  So it is a typical and well tested approach to blocking in one
call while still handling other asynchronous events.  However, the Gtk
device again only knows about its event loop, namely Gtk+.  As a
result, no other event sources will be processed during a call
to \SFunction{locator}. For example, any Tclk/Tk GUI will become
inactive and not even update its display.


\section{Integratin with R}
Currently, this functionality is provided as a package.  This means
that when we start the new event loop, we do not actually return from
the \Croutine{.C} call that performs the initiation.  Instead, this
emits an identical prompt as the R event loop would and then it takes
over. This means that, within the low-level native code, we don't
return from the \Croutine{do_dotCode}.

In the longer term, we would like to make this more seamless.  It
should be relatively easy to move the event loop structure into R
itself and have the call to initialize the new event loop merely
replace the event loop structure with a new collection of routines.
The actual loop will continue in the next iteration but using these
new routines.


\section{Notes}
One can tire gdb out when testing this.  But I have seen this happen
in other circumstances also, so it may not be an inherent property of
this setup.

Issues about running two event loops.  How do we avoid re-registering
the same input/event sources. Do we chain so that we process event
loop 1, then 2, etc. until all are finished and then continue.
Or do we just run the top-most one in the stack.

Error handling and unregistering events.

Signals.

For more sophisticated features (e.g. some provided by Tcl's event
loop such as prepare-to-block initialization of event sources, etc.),
one should use the Tcl event loop. In other words, we don't have to
provide this in full generality since users can swap in a Tcl event
loop and use that. Obviously this isn't as good as handling it in
general since there may be a conflict of interest in a session that
prefers to use two different loop implementations.  For the forseeable
future, this is not a high priority.

Using the R event loop structure, one should be able to create a
hybrid loop. Need classes of events ideally, or specifically
to target input handlers to a particular loop type.


Use the event loop associated with the toolkit which implements the
primary UI element. For example, if we are running the Gnome
interface, use Gtk's event loop. If Tcl/Tk is the preferred toolkit we
use for creating GUIs, use that.

\section{Examples and Tests}

\subsection{Tcl/Tk as a Gtk event source}

Perhaps we can simply add the other event loop's ``do one event''
routine as an idle handler to the main event loop.

\begin{verbatim}
library(tcltk)
dyn.load("gtkReader.so")
source("tk.S")
invisible(.C("R_mainLoop"))
.Call("R_addTclTkToGtk")
library(Rggobi)
ggobi("/home/duncan/Projects/ggobi/ggobi-cvs/data/flea.xml")
\end{verbatim}

Both systems should be interactive and work correctly.



\subsection{GGobi tour and the stand-alone Gtk device's locator}
This example shows GGobi running
with the 
\begin{verbatim}
dyn.load("gtkReader.so")
invisible(.C("R_mainLoop"))
library(gtkDevice)
library(Rggobi)
ggobi("/home/duncan/Projects/ggobi/ggobi-cvs/data/flea.xml")

gtk()
plot(rnorm(100))
locator(n = 3)
\end{verbatim}

Turning on the tour (manually at present) seems to swamp the events
and the end button click events don't get through (?)


\section{Readers and Monitors}
It is useful to be able to watch for input on different connections
and to have an S function be called when input becomes available.
This is the concept of a reader in S4 and similar to facilities in
numerous other environments. For example, in Gtk these are ``input''
sources.  

Similarly, it is useful to be able to schedule calls to S functions to
happen after a certain period of time, or every $t$ time units.  In
S4, such timed actions are called monitors. In Gtk, these are
timeouts.

These two styles of asynchronous events.  Rather than adding a
facility to R to register them and implement them in the absence of an
external event loop (e.g. Tcl/Tk or Gtk), it is simplest to use the
interfaces to those external systems to add such handlers.
Specifically, one can use \SFunction{gtkTimeoutAdd} from the RGtk
package.  We already handle input sources in the R event loop on Unix,
so we can provide an interface to this.  Since we have to get the file
descriptor associated with an R connection object, this is also
reasonable to do in R.


\section{Additional Advantanges}

We can split out the readline functionality from core and allow people
to add it on demand.  This means that they can recompile just a single
package.

The framework makes it easier to add a new event loop engine (e.g.
wxWindows, etc.)



\section{To Do}

\begin{description}
\item[X11 device locator.]  This is tricky because of the lack of
  callbacks.  We can use the centralized event loop global variable
  rather than handleEvent() and that will use the regular event loop.
  We can also have a variable in the X11 event loop for 
  holding the current XEvent and mimicing the callback mechanism
  ourselves.
  Or, we can suspend (or insert earlier in the callback queue) the 
  standard 

\item[Tcl/Tk event loop integration.]
Seems basically okay  with the methods.
Need to get the signatures accurately defined.

\item[Mixing tcl and GTk.]

  Use idle or timeout handlers to put one on the other is pretty much
  working.  It would be good to add a mechanism that smooths out the
  differences between the two systems regarding the way timers and
  idles are automatically re-registered if they return non-zero.

\item Continuations in the prompt and not keeping the initial text around.
(e.g.
\begin{verbatim}
plot(1:10
)
\end{verbatim}
)
This is now fixed.

\item[Merge with existing R code.]

Use \verb+eloop->addInput+ for implementation of
\Croutine{AddInputHandler}.

In the long run, have an event loop pointer in R's
main and use the routines for that.

\item[Generic event loop structure] An initial version is in
  Reventloop.h and can be put into \file{eventloop.h} in
  \dir{include/R_ext}.


\item[] Orbacus and other CORBA event loops.

\item[] error in the browser() and readline 
when debug a callback in Gtk.

This is a general problem with asynchonous calls to browser and
anything that uses the standard console reading facilties
asynchronously.  The basic problem is as follows.  We are in the usual
input loop awaiting characters typed by the user.  Then
asynchronously, we enter the browser due to a callback that is invoked
from the background event loop that is active while waiting for the
user input.  At this point, we essentially are starting a new readline
session and it is important that we restore the old one when we
complete the browse-related one. But unfortunately, we are using
global variables and restoring it is not currently being done.

Cleaning up after errors is still an issue that needs
investigation.

\item[] Error in scan means we get two prompts: the one from returning
  from scan and then another from falling through the contexts up to
  the new event loop's version.

\item[] The GGobi tour dominates Tcl/Tk as an idle task. Perhaps we
  can put the Tcl/Tk event loop as a timeout on the Gtk event loop.
  Same problem with the gtk device and locator when the tour is running.

\end{description}
\end{document} 


The motivation for some of this is to be able to embed an existing
Tcl/Tk or Gtk application directly into R without needing to modify it
to handle, e.g., timer or idle procedures for handling GGobi's tour,
Gtk tooltips, Tcl/Tk's external events.  Similarly, when we embed R
within other applications such as Gnumeric (Gtk), ns (Tcl/Tk), we also
want \SFunction{setMonitor} and \SFunction{setReader} to behave as
expected within those environments.  By using event loops from other
environments (e.g. Gtk, Tcl/Tk, wxWindows), we get a much smoother
integration of those environments and things will work with fewer
surprises. Code written for those systems will behave as expected when
run with R.  Additionally, we might elect only to support timers and
readers on connections in R if the user is deploying another event
mechanism.

In this note, we first describe steps to take-over the R event loop
while still providing the same functionality, but with background
facilities in a GUI (e.g. tooltips, progress bars, etc.) always active
while no long-running task is active.  We will extend this to show how
one can integrate this into the current R framework with minimal
changes. Then we illustrate how one can support multiple event loops
both in type (Tcl/Tk or Gtk) and simultaneously.  In practice, one
will hopefully have a single event loop, but one can run two
simultaneously under this model.  Finally, we describe changes to the
way certain R functions are implemented so that they become event
driven rather than artificially blocking. Functions such as
\SFunction{Sys.sleep} and \SFunction{locator} are used to exemplify
this style of programming.
